\capitulo{2}{Objetivos del proyecto}

A continuación, se detallan los diferentes objetivos que han motivado la realización del proyecto.
\section{Objetivos generales}
\begin{itemize}

	\item Desarrollar una aplicación web que permita obtener un archivo DXF a partir de los datos de un levantamiento topográfico.

	\item Definir un tipo de codificación para los puntos medidos en campo, que sea interpretada por la aplicación, automatice el proceso de dibujo y mejore el rendimiento en el trabajo de campo.

	\item Permitir al usuario crear una cuenta.

	\item Permitir al usuario subir :  archivos de campo, archivos de configuración personalizados de la transformación y archivos DXF, con símbolos creados por el usuario.

	\item Permitir al usuario modificar o elegir, en la interfaz, los nombres de las capas de CAD, colores y símbolos existentes, y asociarlos a los códigos de campo

	\item Permitir al usuario elegir la versión de CAD para generar el archivo DXF.

	\item Permitir al usuario almacenar en su equipo, el archivo DXF generado.
\end{itemize}
\section{Objetivos técnicos}

\begin{itemize}
\item Desarrollar una aplicación web utilizando el framework \textbf {Flask}.

\item Utilizar \textbf {PLY}\footnote{\textsl{PLY}: \url{https://pypi.org/project/ply/}}   (Python Lex-Yacc), como analizador sintáctico, para reconocer la validez de los archivos de entrada.

\item Utilizar \textbf {ezdxf}\footnote{\textsl{ezdf}: \url{https://ezdxf.readthedocs.io/en/latest/index.html}}  ,  para leer, modificar y crear archivos DXF. 

\item Utilizar \textbf {Bootstrap}, para crear el sitio web.

\item Utilizar \textbf {TinyColor}\footnote{\textsl{TinyColor}: \url{https://github.com/bgrins/TinyColor}} para poder definir los colores con los estándares de AutoCAD.

\item Utilizar  \textbf {PostGIS}\footnote{\textsl{PostGIS}: \url{https://postgis.net/}} como sistema de base de datos.

\item Realizar test unitarios.

\item Hacer uso de la herramienta \textbf {CODEBEAT} para comprobar la calidad del código.

\item Desplegar la aplicación usando \textbf {Docker} .

\item Utilizar \textbf {Git} como sistema de control de versiones distribuido junto con la plataforma \textbf {GitHub}.

\item Aplicar la metodología ágil \textbf {Scrum} para el desarrollo del software.

\item Utilizar \textbf {ZenHub} como herramienta de gestión de proyectos.
\end{itemize}

\section{Objetivos personales}
\begin{itemize}
\item Mejorar los recursos invertidos y facilitar el trabajo, a la hora de realizar un levantamiento topográfico.

\item Realizar el proyecto como un trabajo real, aplicando los conocimientos adquiridos en la realización del Grado.

\item Utilizar herramientas y metodologías demandadas en el mercado laboral; Git, desarrollo Web, Docker, etc.
\end{itemize}