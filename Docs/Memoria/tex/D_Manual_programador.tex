\apendice{Documentación técnica de programación}

\section{Introducción}
En este apartado se detalla la documentación técnica de programación, estructura de la aplicación, instalación de la aplicación, test realizados, etc. Datos necesarios que pueden servir de guía a cualquier programador pretenda continuar con este proyecto.

\section{Estructura de directorios}

\emph{Flask} permite una gran libertad a la hora de estructurar un proyecto, aunque por convención sigue una estructura determinada. Esta aplicación ha seguido esa estructura. A continuación se expone:

\begin{itemize}

\item \texttt{\textbf{SurveyingPointCode/:}} directorio raíz de la aplicación que contiene archivos como: \texttt{surveyingpointcode.py} y \texttt{config.py}, que son de inicio y configuración de la aplicación, respectivamente. También se encuentra un archivo con los requerimientos necesarios \texttt{requirements.txt} y un archivo con las variables de entorno que se deben cargar al sistema \texttt{app-env}. Por otra parte, se incluyen los archivos necesarios para el despliegue con \emph{Docker}, estos de detallarán más adelante.

\item \texttt{\textbf{SurveyingPointCode/app/:}} contiene todos los archivos que definen la lógica del negocio. 

\item \texttt{\textbf{SurveyingPointCode/app/templates/:}} contiene todos los archivos con las plantillas \emph{ html}. 

\item \texttt{\textbf{SurveyingPointCode/app/static/:}} contiene todos los archivos estáticos.

\item \texttt{\textbf{SurveyingPointCode/app/static/css:}} contiene todos los archivos con los estilos.

\item \texttt{\textbf{SurveyingPointCode/app/static/js:}} contiene todos los archivos \emph{JavaScript}.

\item \texttt{\textbf{SurveyingPointCode/app/static/images/:}} contiene todos los archivos con las imágenes de la que se usan en la aplicación Web.

\item \texttt{\textbf{SurveyingPointCode/app/static/webfonts/:}} contiene todos los archivos con las fuentes de la que se usan en la aplicación Web.

\item \texttt{\textbf{SurveyingPointCode/tmp/:}} contiene todos los archivos temporales que se generan en cada sesión, tanto los archivos subidos por el usuario, como las conversiones.

\item \texttt{\textbf{Archivos\_Prueba/:}} contiene diferentes archivos de ejemplo para cargar en la aplicación, tanto válidos como con errores.

\item \texttt{\textbf{Archivos\_Prueba/Datos\_de\_campo/:}} contiene diferentes archivos de campo, procedentes de un levantamiento topográfico.

\item \texttt{\textbf{Archivos\_Prueba/Configuración\_usuario/:}} contiene diferentes archivos de configuración de la conversión.

\item \texttt{\textbf{Archivos\_Prueba/Simbolos/:}} contiene diferentes archivos DXF, algunos válidos con símbolos definidos y otros vacíos.

\item \texttt{\textbf{Test\_unitarios/:}} contiene diferentes archivos para realizar los test unitarios.

\end{itemize}


\section{Manual del programador}

Este manual tiene como objetivo servir de referencia a futuros programadores que trabajen en la aplicación o a cualquier persona interesada en su construcción. 

Existen dos vías a seguir:

\begin{enumerate}

\item Configuración del entorno de desarrollo en un equipo instalando todas las herramientas necesarias. 

\item Utilizar \emph{Docker} como entorno de desarrollo. 

\end{enumerate}

En este proyecto se han seguido las dos vías. Se llevó a cabo todo el proceso con la primera, hasta que la aplicación estuvo montada completamente en \emph{Docker}, por lo que se siguió con la segunda vía. 
El proyecto se ha desarrollado utilizando un equipo con sistema operativo \emph{Linux} en su distribución \emph{Linux Mint 18.3 Sylvia}. Los comandos a continuación detallados son para ese sistema operativo.

\subsection{Configuración del entorno de desarrollo en un equipo}

Esta es la forma más convencional de realizar un desarrollo, instalar  las herramientas, bibliotecas y módulos en el equipo. A continuación se describen brevemente los pasos seguidos:

\begin{enumerate}

\item En primer lugar descargar el código de la aplicación que se encuentra en \emph{GitHub} o clonar el repositorio al equipo local.

\item Instalación de \emph{Anaconda} con Python 3.6. La opción de usar \emph{Anaconda} es para facilitar la opción de trabajar con entornos virtuales personalizados, sin necesidad de instalar dependencias directamente en nuestro equipo.

\item Creación de un entorno virtual en \emph{Anaconda}.

\texttt{conda create -n nombre\_de\_entorno python=3.6 anaconda}

\item Activar el entorno virtual en \emph{Anaconda}.

\texttt{source activate  nombre\_de\_entorno}

\item Instalar las dependencias dentro del entorno virtual creado. Para ello necesitamos el archivo \texttt{requirements.txt}, mencionado anteriormente.
\texttt{pip install -r requirements.txt}

\texttt Para trabajar con la base de datos \emph{PostGIS}, se utiliza un contenedor \emph{Docker}. En este punto se debe tener instalado \emph{Docker}. El contenedor de \emph{PostGIS} se crea con este comando:

\begin{verbatim}
sudo docker run --name=postgis --hostname=postgres --network=pgnetwork
-d -e POSTGRES_USER=tfg -e POSTGRES_PASS=f04f1b4d7734f0dc3c4da46f19c0a9f49b56
-e POSTGRES_DBNAME=tfg -e ALLOW_IP_RANGE=0.0.0.0/0 
-p 5432:5432 -v pg_data:/var/lib/postgresql --restart=always mdillon/postgis
\end{verbatim}

siendo sus parámetros de conexión con la BBDD:

\begin{verbatim}
user=tfg
password=f04f1b4d7734f0dc3c4da46f19c0a9f49b56
db_name=tfg
port=5432
host=postgis

\end{verbatim}


\end{enumerate}
El entorno de desarrollo está configurado. El \emph{IDE} o editor de texto utilizado es una opción personal del desarrollador, en este proyecto se ha usado \emph{PyCharm}.



\subsection{Utilizar \emph{Docker} como entorno de desarrollo.}

Entre las múltiples ventajas de usar \emph{Docker}, es importante destacar que se puede usar tanto en la etapa de desarrollo como, finalmente en el despliegue. Garantiza que el entorno de desarrollo va a ser siempre el mismo, independientemente del equipo en que se trabaje.

En este proyecto, el uso de \emph{Docker} facilitará la continuidad del desarrollo, solamente hace falta tener instalado en el equipo, la aplicación \emph{Docker}. Siguiendo las instrucciones del archivo \texttt{README.txt} que a continuación se enumeran:

\begin{enumerate}
\item Instalación de \emph{Docker.}
\item Instalación de \emph{Docker Compose.}
\item Clonar el repositorio.
\item Ejecutar en el directorio \texttt{SurveyingPointCode/}:

\texttt{sh app\_install.sh}

Cada vez que realicemos un cambio en el código, debemos actualizar este cambio en el contenedor correspondiente, y lo haremos de la siguiente forma:
\begin{enumerate}

\item Ejecutamos \texttt{docker-compose down} para detener el contendedor.

\item Ejecutamos \texttt{docker-compose up -d} para volver a construir los contenedores y levantarlos.
\end{enumerate}

\end{enumerate}

Los archivos \texttt{Dockerfile} y \texttt{docker-compose.yml}, ya comentados en el documento de la memoria( ver apartado Despliegue de la aplicación — \emph{Docker Compose}.), son la base para la construcción de los contenedores y por consiguiente, de los servicios ofrecidos. 

Como contrapunto, se ha encontrado una desventaja al usar \emph{Docker} como entorno de desarrollo, el tiempo invertido en volver a construir el contenedor. A medida que el proyecto crece y se van aumentando las dependencias, el tiempo de creación del contenedor se alarga. Es conveniente seguir investigando sobre como solucionar este problema.

\section{Compilación, instalación y ejecución del proyecto}\label{sec:instalacion}

Como en el apartado anterior la instalación de la aplicación es sencillo gracias al uso de \emph{Docker} y tambien garantiza que se va a comportar igual independientemente del equipo donde se instale. Siguiendo las instrucciones del archivo \texttt{README.txt} que a continuación se enumeran, se realiza la instalación:

\begin{enumerate}
\item Instalación de \emph{Docker.}
\item Instalación de \emph{Docker Compose.}
\item Clonar el repositorio.
\item Ejecutar en el directorio \texttt{SurveyingPointCode/}:

\begin{itemize}
\item Si se desea solo instalar a aplicación \emph{SurveyingPointCode}:

\texttt{sh app\_install.sh}

\item Si se desea instalar ademas la aplicación \emph{PgAdmin4}:

\texttt{sh app\_install\_pgadmin.sh}

\end{itemize}

\item Acceder en el navegador a la dirección \url{http://0.0.0.0:5000} para \emph{SurveyingPointCode}:

\item Acceder en el navegador a la dirección \url{http://0.0.0.0:80} para \emph{PgAdmin 4}:

\end{enumerate}


\section{Pruebas del sistema}

Para comprobar el funcionamiento de las funciones principales de la aplicación se han realizado una serie de test unitarios. 

Se ha utilizado la biblioteca de python \texttt{unittest}.

\newpage
\subsection{Caso de prueba CP-01}

\textbf{Descripción:} Comprueba si el número de capas creadas en un modelo de dibujo es correcto una vez cargado y procesado un archivo de campo, con la función \texttt{create\_layers()}.

\textbf{Datos de entrada:} \texttt{Example\_topographic\_1.txt}


\begin{longtable}[]{@{}llll@{}}
\toprule
\begin{minipage}[b]{0.6\columnwidth}\raggedright\strut
Función\strut
\end{minipage} & \begin{minipage}[b]{0.20\columnwidth}\raggedright\strut
Condición\strut
\end{minipage} & \begin{minipage}[b]{0.15\columnwidth}\raggedright\strut
Salida Esperada\strut
\end{minipage} & \begin{minipage}[b]{0.05\columnwidth}\raggedright\strut
Ok\strut
\end{minipage}\tabularnewline
\midrule
\endhead
\begin{minipage}[t]{0.6\columnwidth}\raggedright\strut
test\_create\_layers\_number()\strut
\end{minipage} & \begin{minipage}[t]{0.20\columnwidth}\raggedright\strut
¿Es igual?\strut
\end{minipage} & \begin{minipage}[t]{0.15\columnwidth}\raggedright\strut
8\strut
\end{minipage} & \begin{minipage}[t]{0.05\columnwidth}\raggedright\strut
Si\strut
\end{minipage}\tabularnewline
\begin{minipage}[t]{0.6\columnwidth}\raggedright\strut
\strut
\end{minipage} & \begin{minipage}[t]{0.20\columnwidth}\raggedright\strut
¿No es igual?\strut
\end{minipage} & \begin{minipage}[t]{0.15\columnwidth}\raggedright\strut
0\strut
\end{minipage} & \begin{minipage}[t]{0.05\columnwidth}\raggedright\strut
Si\strut
\end{minipage}\tabularnewline

\bottomrule
\caption{Caso de prueba CP-01.}
\end{longtable}

\subsection{Caso de prueba CP-02}

\textbf{Descripción:} Comprueba el nombre de las capas creadas es correcto una vez cargado y procesado un archivo de campo, con la función \texttt{create\_layers()}.

\textbf{Datos de entrada:} \texttt{Example\_topographic\_1.txt}


\begin{longtable}[]{@{}llll@{}}
\toprule
\begin{minipage}[b]{0.6\columnwidth}\raggedright\strut
Función\strut
\end{minipage} & \begin{minipage}[b]{0.20\columnwidth}\raggedright\strut
Condición\strut
\end{minipage} & \begin{minipage}[b]{0.15\columnwidth}\raggedright\strut
Salida Esperada\strut
\end{minipage} & \begin{minipage}[b]{0.05\columnwidth}\raggedright\strut
Ok\strut
\end{minipage}\tabularnewline
\midrule
\endhead
\begin{minipage}[t]{0.6\columnwidth}\raggedright\strut
test\_create\_layers\_types()\strut
\end{minipage} & \begin{minipage}[t]{0.20\columnwidth}\raggedright\strut
¿Contiene?\strut
\end{minipage} & \begin{minipage}[t]{0.15\columnwidth}\raggedright\strut
Si\strut
\end{minipage} & \begin{minipage}[t]{0.05\columnwidth}\raggedright\strut
Si\strut
\end{minipage}\tabularnewline
\bottomrule
\caption{Caso de prueba CP-02.}
\end{longtable}

\subsection{Caso de prueba CP-03}

\textbf{Descripción:} Comprueba si el número de puntos insertados en un modelo de dibujo es correcto una vez cargado y procesado un archivo de campo, con la función \texttt{get\_points()}.

\textbf{Datos de entrada:} \texttt{Example\_topographic\_1.txt}


\begin{longtable}[]{@{}llll@{}}
\toprule
\begin{minipage}[b]{0.6\columnwidth}\raggedright\strut
Función\strut
\end{minipage} & \begin{minipage}[b]{0.20\columnwidth}\raggedright\strut
Condición\strut
\end{minipage} & \begin{minipage}[b]{0.15\columnwidth}\raggedright\strut
Salida Esperada\strut
\end{minipage} & \begin{minipage}[b]{0.05\columnwidth}\raggedright\strut
Ok\strut
\end{minipage}\tabularnewline
\midrule
\endhead
\begin{minipage}[t]{0.6\columnwidth}\raggedright\strut
test\_create\_points\_number\_file\_correct()\strut
\end{minipage} & \begin{minipage}[t]{0.20\columnwidth}\raggedright\strut
¿Es igual?\strut
\end{minipage} & \begin{minipage}[t]{0.15\columnwidth}\raggedright\strut
42\strut
\end{minipage} & \begin{minipage}[t]{0.05\columnwidth}\raggedright\strut
Si\strut
\end{minipage}\tabularnewline
\begin{minipage}[t]{0.6\columnwidth}\raggedright\strut
\strut
\end{minipage} & \begin{minipage}[t]{0.20\columnwidth}\raggedright\strut
¿No es igual?\strut
\end{minipage} & \begin{minipage}[t]{0.15\columnwidth}\raggedright\strut
0\strut
\end{minipage} & \begin{minipage}[t]{0.05\columnwidth}\raggedright\strut
Si\strut
\end{minipage}\tabularnewline

\bottomrule
\caption{Caso de prueba CP-03.}
\end{longtable}

\subsection{Caso de prueba CP-04}

\textbf{Descripción:} Comprueba si el número de textos insertados en un modelo de dibujo es correcto una vez cargado y procesado un archivo de campo, con la función \texttt{get\_points()}.

\textbf{Datos de entrada:} \texttt{Example\_topographic\_1.txt}


\begin{longtable}[]{@{}llll@{}}
\toprule
\begin{minipage}[b]{0.6\columnwidth}\raggedright\strut
Función\strut
\end{minipage} & \begin{minipage}[b]{0.20\columnwidth}\raggedright\strut
Condición\strut
\end{minipage} & \begin{minipage}[b]{0.15\columnwidth}\raggedright\strut
Salida Esperada\strut
\end{minipage} & \begin{minipage}[b]{0.05\columnwidth}\raggedright\strut
Ok\strut
\end{minipage}\tabularnewline
\midrule
\endhead
\begin{minipage}[t]{0.6\columnwidth}\raggedright\strut
test\_create\_points\_texts\_file\_correct()\strut
\end{minipage} & \begin{minipage}[t]{0.20\columnwidth}\raggedright\strut
¿Es igual?\strut
\end{minipage} & \begin{minipage}[t]{0.15\columnwidth}\raggedright\strut
126\strut
\end{minipage} & \begin{minipage}[t]{0.05\columnwidth}\raggedright\strut
Si\strut
\end{minipage}\tabularnewline
\begin{minipage}[t]{0.6\columnwidth}\raggedright\strut
\strut
\end{minipage} & \begin{minipage}[t]{0.20\columnwidth}\raggedright\strut
¿No es igual?\strut
\end{minipage} & \begin{minipage}[t]{0.15\columnwidth}\raggedright\strut
0\strut
\end{minipage} & \begin{minipage}[t]{0.05\columnwidth}\raggedright\strut
Si\strut
\end{minipage}\tabularnewline

\bottomrule
\caption{Caso de prueba CP-04.}
\end{longtable}

\subsection{Caso de prueba CP-05}

\textbf{Descripción:} Comprueba si el número de círculos insertados en un modelo de dibujo es correcto una vez cargado y procesado un archivo de campo, con la función \texttt{get\_circles()}.

\textbf{Datos de entrada:} \texttt{Example\_topographic\_1.txt}


\begin{longtable}[]{@{}llll@{}}
\toprule
\begin{minipage}[b]{0.6\columnwidth}\raggedright\strut
Función\strut
\end{minipage} & \begin{minipage}[b]{0.20\columnwidth}\raggedright\strut
Condición\strut
\end{minipage} & \begin{minipage}[b]{0.15\columnwidth}\raggedright\strut
Salida Esperada\strut
\end{minipage} & \begin{minipage}[b]{0.05\columnwidth}\raggedright\strut
Ok\strut
\end{minipage}\tabularnewline
\midrule
\endhead
\begin{minipage}[t]{0.6\columnwidth}\raggedright\strut
test\_create\_circles\_number()\strut
\end{minipage} & \begin{minipage}[t]{0.20\columnwidth}\raggedright\strut
¿Es igual?\strut
\end{minipage} & \begin{minipage}[t]{0.15\columnwidth}\raggedright\strut
2\strut
\end{minipage} & \begin{minipage}[t]{0.05\columnwidth}\raggedright\strut
Si\strut
\end{minipage}\tabularnewline
\begin{minipage}[t]{0.6\columnwidth}\raggedright\strut
\strut
\end{minipage} & \begin{minipage}[t]{0.20\columnwidth}\raggedright\strut
¿No es igual?\strut
\end{minipage} & \begin{minipage}[t]{0.15\columnwidth}\raggedright\strut
0\strut
\end{minipage} & \begin{minipage}[t]{0.05\columnwidth}\raggedright\strut
Si\strut
\end{minipage}\tabularnewline

\bottomrule
\caption{Caso de prueba CP-05.}
\end{longtable}

\subsection{Caso de prueba CP-06}

\textbf{Descripción:} Comprueba si el número de \emph{splines} insertados en un modelo de dibujo es correcto una vez cargado y procesado un archivo de campo, con la función \texttt{get\_splines()}.

\textbf{Datos de entrada:} \texttt{Example\_topographic\_1.txt}


\begin{longtable}[]{@{}llll@{}}
\toprule
\begin{minipage}[b]{0.6\columnwidth}\raggedright\strut
Función\strut
\end{minipage} & \begin{minipage}[b]{0.20\columnwidth}\raggedright\strut
Condición\strut
\end{minipage} & \begin{minipage}[b]{0.15\columnwidth}\raggedright\strut
Salida Esperada\strut
\end{minipage} & \begin{minipage}[b]{0.05\columnwidth}\raggedright\strut
Ok\strut
\end{minipage}\tabularnewline
\midrule
\endhead
\begin{minipage}[t]{0.6\columnwidth}\raggedright\strut
test\_create\_splines\_number()\strut
\end{minipage} & \begin{minipage}[t]{0.20\columnwidth}\raggedright\strut
¿Es igual?\strut
\end{minipage} & \begin{minipage}[t]{0.15\columnwidth}\raggedright\strut
1\strut
\end{minipage} & \begin{minipage}[t]{0.05\columnwidth}\raggedright\strut
Si\strut
\end{minipage}\tabularnewline
\begin{minipage}[t]{0.6\columnwidth}\raggedright\strut
\strut
\end{minipage} & \begin{minipage}[t]{0.20\columnwidth}\raggedright\strut
¿No es igual?\strut
\end{minipage} & \begin{minipage}[t]{0.15\columnwidth}\raggedright\strut
0\strut
\end{minipage} & \begin{minipage}[t]{0.05\columnwidth}\raggedright\strut
Si\strut
\end{minipage}\tabularnewline

\bottomrule
\caption{Caso de prueba CP-06.}
\end{longtable}

\subsection{Caso de prueba CP-07}

\textbf{Descripción:} Comprueba si el número de líneas insertadas en un modelo de dibujo es correcto una vez cargado y procesado un archivo de campo, con la función \texttt{get\_lines()}.

\textbf{Datos de entrada:} \texttt{Example\_topographic\_1.txt}


\begin{longtable}[]{@{}llll@{}}
\toprule
\begin{minipage}[b]{0.6\columnwidth}\raggedright\strut
Función\strut
\end{minipage} & \begin{minipage}[b]{0.20\columnwidth}\raggedright\strut
Condición\strut
\end{minipage} & \begin{minipage}[b]{0.15\columnwidth}\raggedright\strut
Salida Esperada\strut
\end{minipage} & \begin{minipage}[b]{0.05\columnwidth}\raggedright\strut
Ok\strut
\end{minipage}\tabularnewline
\midrule
\endhead
\begin{minipage}[t]{0.6\columnwidth}\raggedright\strut
test\_create\_lines\_number()\strut
\end{minipage} & \begin{minipage}[t]{0.20\columnwidth}\raggedright\strut
¿Es igual?\strut
\end{minipage} & \begin{minipage}[t]{0.15\columnwidth}\raggedright\strut
5\strut
\end{minipage} & \begin{minipage}[t]{0.05\columnwidth}\raggedright\strut
Si\strut
\end{minipage}\tabularnewline
\begin{minipage}[t]{0.6\columnwidth}\raggedright\strut
\strut
\end{minipage} & \begin{minipage}[t]{0.20\columnwidth}\raggedright\strut
¿No es igual?\strut
\end{minipage} & \begin{minipage}[t]{0.15\columnwidth}\raggedright\strut
0\strut
\end{minipage} & \begin{minipage}[t]{0.05\columnwidth}\raggedright\strut
Si\strut
\end{minipage}\tabularnewline

\bottomrule
\caption{Caso de prueba CP-07.}
\end{longtable}

\subsection{Caso de prueba CP-08}

\textbf{Descripción:} Comprueba si el número de cuadrados insertados en un modelo de dibujo es correcto una vez cargado y procesado un archivo de campo, con la función \texttt{get\_squares()}.

\textbf{Datos de entrada:} \texttt{Example\_topographic\_1.txt}


\begin{longtable}[]{@{}llll@{}}
\toprule
\begin{minipage}[b]{0.6\columnwidth}\raggedright\strut
Función\strut
\end{minipage} & \begin{minipage}[b]{0.20\columnwidth}\raggedright\strut
Condición\strut
\end{minipage} & \begin{minipage}[b]{0.15\columnwidth}\raggedright\strut
Salida Esperada\strut
\end{minipage} & \begin{minipage}[b]{0.05\columnwidth}\raggedright\strut
Ok\strut
\end{minipage}\tabularnewline
\midrule
\endhead
\begin{minipage}[t]{0.6\columnwidth}\raggedright\strut
test\_create\_squares\_number()\strut
\end{minipage} & \begin{minipage}[t]{0.20\columnwidth}\raggedright\strut
¿Es igual?\strut
\end{minipage} & \begin{minipage}[t]{0.15\columnwidth}\raggedright\strut
5\strut
\end{minipage} & \begin{minipage}[t]{0.05\columnwidth}\raggedright\strut
Si\strut
\end{minipage}\tabularnewline
\begin{minipage}[t]{0.6\columnwidth}\raggedright\strut
\strut
\end{minipage} & \begin{minipage}[t]{0.20\columnwidth}\raggedright\strut
¿No es igual?\strut
\end{minipage} & \begin{minipage}[t]{0.15\columnwidth}\raggedright\strut
0\strut
\end{minipage} & \begin{minipage}[t]{0.05\columnwidth}\raggedright\strut
Si\strut
\end{minipage}\tabularnewline

\bottomrule
\caption{Caso de prueba CP-08.}
\end{longtable}

\subsection{Caso de prueba CP-09}

\textbf{Descripción:} Comprueba si el número de rectángulos insertados en un modelo de dibujo es correcto una vez cargado y procesado un archivo de campo, con la función \texttt{get\_rectangles()}.

\textbf{Datos de entrada:} \texttt{Example\_topographic\_1.txt}


\begin{longtable}[]{@{}llll@{}}
\toprule
\begin{minipage}[b]{0.6\columnwidth}\raggedright\strut
Función\strut
\end{minipage} & \begin{minipage}[b]{0.20\columnwidth}\raggedright\strut
Condición\strut
\end{minipage} & \begin{minipage}[b]{0.15\columnwidth}\raggedright\strut
Salida Esperada\strut
\end{minipage} & \begin{minipage}[b]{0.05\columnwidth}\raggedright\strut
Ok\strut
\end{minipage}\tabularnewline
\midrule
\endhead
\begin{minipage}[t]{0.6\columnwidth}\raggedright\strut
test\_create\_squares\_number()\strut
\end{minipage} & \begin{minipage}[t]{0.20\columnwidth}\raggedright\strut
¿Es igual?\strut
\end{minipage} & \begin{minipage}[t]{0.15\columnwidth}\raggedright\strut
1\strut
\end{minipage} & \begin{minipage}[t]{0.05\columnwidth}\raggedright\strut
Si\strut
\end{minipage}\tabularnewline
\begin{minipage}[t]{0.6\columnwidth}\raggedright\strut
\strut
\end{minipage} & \begin{minipage}[t]{0.20\columnwidth}\raggedright\strut
¿No es igual?\strut
\end{minipage} & \begin{minipage}[t]{0.15\columnwidth}\raggedright\strut
0\strut
\end{minipage} & \begin{minipage}[t]{0.05\columnwidth}\raggedright\strut
Si\strut
\end{minipage}\tabularnewline

\bottomrule
\caption{Caso de prueba CP-09.}
\end{longtable}

\subsection{Caso de prueba CP-10}

\textbf{Descripción:} Comprueba que un archivo sin círculos no crea ningún círculo en un modelo de dibujo una vez cargado y procesado un archivo de campo, con la función \texttt{get\_circles()}.

\textbf{Datos de entrada:} \texttt{Example\_topographic\_2.txt}


\begin{longtable}[]{@{}llll@{}}
\toprule
\begin{minipage}[b]{0.6\columnwidth}\raggedright\strut
Función\strut
\end{minipage} & \begin{minipage}[b]{0.20\columnwidth}\raggedright\strut
Condición\strut
\end{minipage} & \begin{minipage}[b]{0.15\columnwidth}\raggedright\strut
Salida Esperada\strut
\end{minipage} & \begin{minipage}[b]{0.05\columnwidth}\raggedright\strut
Ok\strut
\end{minipage}\tabularnewline
\midrule
\endhead
\begin{minipage}[t]{0.6\columnwidth}\raggedright\strut
test\_not\_create\_circles()\strut
\end{minipage} & \begin{minipage}[t]{0.20\columnwidth}\raggedright\strut
¿Es igual?\strut
\end{minipage} & \begin{minipage}[t]{0.15\columnwidth}\raggedright\strut
0\strut
\end{minipage} & \begin{minipage}[t]{0.05\columnwidth}\raggedright\strut
Si\strut
\end{minipage}\tabularnewline

\bottomrule
\caption{Caso de prueba CP-10.}
\end{longtable}

\subsection{Caso de prueba CP-11}

\textbf{Descripción:} Comprueba que un archivo sin \emph{splines} no crea ninguna \emph{spline} en un modelo de dibujo una vez cargado y procesado un archivo de campo, con la función \texttt{get\_splines()}.

\textbf{Datos de entrada:} \texttt{Example\_topographic\_2.txt}


\begin{longtable}[]{@{}llll@{}}
\toprule
\begin{minipage}[b]{0.6\columnwidth}\raggedright\strut
Función\strut
\end{minipage} & \begin{minipage}[b]{0.20\columnwidth}\raggedright\strut
Condición\strut
\end{minipage} & \begin{minipage}[b]{0.15\columnwidth}\raggedright\strut
Salida Esperada\strut
\end{minipage} & \begin{minipage}[b]{0.05\columnwidth}\raggedright\strut
Ok\strut
\end{minipage}\tabularnewline
\midrule
\endhead
\begin{minipage}[t]{0.6\columnwidth}\raggedright\strut
test\_not\_create\_splines()\strut
\end{minipage} & \begin{minipage}[t]{0.20\columnwidth}\raggedright\strut
¿Es igual?\strut
\end{minipage} & \begin{minipage}[t]{0.15\columnwidth}\raggedright\strut
0\strut
\end{minipage} & \begin{minipage}[t]{0.05\columnwidth}\raggedright\strut
Si\strut
\end{minipage}\tabularnewline

\bottomrule
\caption{Caso de prueba CP-11.}
\end{longtable}

\subsection{Caso de prueba CP-12}

\textbf{Descripción:} Comprueba que un archivo sin líneas no crea ninguna línea en un modelo de dibujo una vez cargado y procesado un archivo de campo, con la función \texttt{get\_lines()}.

\textbf{Datos de entrada:} \texttt{Example\_topographic\_2.txt}


\begin{longtable}[]{@{}llll@{}}
\toprule
\begin{minipage}[b]{0.6\columnwidth}\raggedright\strut
Función\strut
\end{minipage} & \begin{minipage}[b]{0.20\columnwidth}\raggedright\strut
Condición\strut
\end{minipage} & \begin{minipage}[b]{0.15\columnwidth}\raggedright\strut
Salida Esperada\strut
\end{minipage} & \begin{minipage}[b]{0.05\columnwidth}\raggedright\strut
Ok\strut
\end{minipage}\tabularnewline
\midrule
\endhead
\begin{minipage}[t]{0.6\columnwidth}\raggedright\strut
test\_not\_create\_lines()\strut
\end{minipage} & \begin{minipage}[t]{0.20\columnwidth}\raggedright\strut
¿Es igual?\strut
\end{minipage} & \begin{minipage}[t]{0.15\columnwidth}\raggedright\strut
0\strut
\end{minipage} & \begin{minipage}[t]{0.05\columnwidth}\raggedright\strut
Si\strut
\end{minipage}\tabularnewline

\bottomrule
\caption{Caso de prueba CP-12.}
\end{longtable}


\subsection{Caso de prueba CP-13}

\textbf{Descripción:} Comprueba que un archivo sin cuadrados ni rectángulos, no los crea en un modelo de dibujo una vez cargado y procesado un archivo de campo, con la función \texttt{get\_lines()}.

\textbf{Datos de entrada:} \texttt{Example\_topographic\_2.txt}


\begin{longtable}[]{@{}llll@{}}
\toprule
\begin{minipage}[b]{0.6\columnwidth}\raggedright\strut
Función\strut
\end{minipage} & \begin{minipage}[b]{0.20\columnwidth}\raggedright\strut
Condición\strut
\end{minipage} & \begin{minipage}[b]{0.15\columnwidth}\raggedright\strut
Salida Esperada\strut
\end{minipage} & \begin{minipage}[b]{0.05\columnwidth}\raggedright\strut
Ok\strut
\end{minipage}\tabularnewline
\midrule
\endhead
\begin{minipage}[t]{0.6\columnwidth}\raggedright\strut
test\_not\_create\_squares\_rectangles()\strut
\end{minipage} & \begin{minipage}[t]{0.20\columnwidth}\raggedright\strut
¿Es igual?\strut
\end{minipage} & \begin{minipage}[t]{0.15\columnwidth}\raggedright\strut
0\strut
\end{minipage} & \begin{minipage}[t]{0.05\columnwidth}\raggedright\strut
Si\strut
\end{minipage}\tabularnewline

\bottomrule
\caption{Caso de prueba CP-13.}
\end{longtable}

\subsection{Caso de prueba CP-14}

\textbf{Descripción:} Comprueba que use calculan correctamente el acimut y la distancia entre dos puntos, a y b, con la función \texttt{calculate\_azimut\_distance()}

\textbf{Datos de entrada:} \texttt{a = [1, (0, 0, 0), 'E'] y  b = [2, (100, 100, 0), 'E']}


\begin{longtable}[]{@{}llll@{}}
\toprule
\begin{minipage}[b]{0.6\columnwidth}\raggedright\strut
Función\strut
\end{minipage} & \begin{minipage}[b]{0.20\columnwidth}\raggedright\strut
Condición\strut
\end{minipage} & \begin{minipage}[b]{0.15\columnwidth}\raggedright\strut
Salida Esperada\strut
\end{minipage} & \begin{minipage}[b]{0.05\columnwidth}\raggedright\strut
Ok\strut
\end{minipage}\tabularnewline
\midrule
\endhead
\begin{minipage}[t]{0.6\columnwidth}\raggedright\strut
test\_azimut\_distance()\strut
\end{minipage} & \begin{minipage}[t]{0.20\columnwidth}\raggedright\strut
¿Es igual?\strut
\end{minipage} & \begin{minipage}[t]{0.15\columnwidth}\raggedright\strut
az=45º dist=141.4213\strut
\end{minipage} & \begin{minipage}[t]{0.05\columnwidth}\raggedright\strut
Si\strut
\end{minipage}\tabularnewline
\begin{minipage}[t]{0.6\columnwidth}\raggedright\strut
\strut
\end{minipage} & \begin{minipage}[t]{0.20\columnwidth}\raggedright\strut
¿No es igual?\strut
\end{minipage} & \begin{minipage}[t]{0.15\columnwidth}\raggedright\strut
az=50º dist=145\strut
\end{minipage} & \begin{minipage}[t]{0.05\columnwidth}\raggedright\strut
Si\strut
\end{minipage}\tabularnewline

\bottomrule
\caption{Caso de prueba CP-14.}
\end{longtable}



\subsection{Caso de prueba CP-15}

\textbf{Descripción:} Comprueba que se calculan correctamente el incremento de $X$ y de $Y$, entre dos puntos, conociendo el acimut y la distancia entre ellos, con la función \texttt{calculate\_increment\_x\_y()}

\textbf{Datos de entrada:} \texttt{az = 45º y dist = 141.4213562373095}
\newpage

\begin{longtable}[]{@{}llll@{}}
\toprule
\begin{minipage}[b]{0.6\columnwidth}\raggedright\strut
Función\strut
\end{minipage} & \begin{minipage}[b]{0.20\columnwidth}\raggedright\strut
Condición\strut
\end{minipage} & \begin{minipage}[b]{0.15\columnwidth}\raggedright\strut
Salida Esperada\strut
\end{minipage} & \begin{minipage}[b]{0.05\columnwidth}\raggedright\strut
Ok\strut
\end{minipage}\tabularnewline
\midrule
\endhead
\begin{minipage}[t]{0.6\columnwidth}\raggedright\strut
test\_increment\_x\_y()\strut
\end{minipage} & \begin{minipage}[t]{0.20\columnwidth}\raggedright\strut
¿Es igual?\strut
\end{minipage} & \begin{minipage}[t]{0.15\columnwidth}\raggedright\strut
In\_x=100 In\_y=100\strut
\end{minipage} & \begin{minipage}[t]{0.05\columnwidth}\raggedright\strut
Si\strut
\end{minipage}\tabularnewline
\begin{minipage}[t]{0.6\columnwidth}\raggedright\strut
\strut
\end{minipage} & \begin{minipage}[t]{0.20\columnwidth}\raggedright\strut
¿No es igual?\strut
\end{minipage} & \begin{minipage}[t]{0.15\columnwidth}\raggedright\strut
In\_x=150 In\_y=145\strut
\end{minipage} & \begin{minipage}[t]{0.05\columnwidth}\raggedright\strut
Si\strut
\end{minipage}\tabularnewline

\bottomrule
\caption{Caso de prueba CP-15.}
\end{longtable}


\subsection{Caso de prueba CP-16}

\textbf{Descripción:} Comprueba que se calculan correctamente un angulo, introduciendo un angulo y una distancia positiva o negativa, con la función \texttt{calculate\_angle()}

\textbf{Datos de entrada:} \texttt{az = 125º y dist =-10}


\begin{longtable}[]{@{}llll@{}}
\toprule
\begin{minipage}[b]{0.6\columnwidth}\raggedright\strut
Función\strut
\end{minipage} & \begin{minipage}[b]{0.20\columnwidth}\raggedright\strut
Condición\strut
\end{minipage} & \begin{minipage}[b]{0.15\columnwidth}\raggedright\strut
Salida Esperada\strut
\end{minipage} & \begin{minipage}[b]{0.05\columnwidth}\raggedright\strut
Ok\strut
\end{minipage}\tabularnewline
\midrule
\endhead
\begin{minipage}[t]{0.6\columnwidth}\raggedright\strut
test\_angle\_direction()\strut
\end{minipage} & \begin{minipage}[t]{0.20\columnwidth}\raggedright\strut
¿Es igual?\strut
\end{minipage} & \begin{minipage}[t]{0.15\columnwidth}\raggedright\strut
35º\strut
\end{minipage} & \begin{minipage}[t]{0.05\columnwidth}\raggedright\strut
Si\strut
\end{minipage}\tabularnewline
\begin{minipage}[t]{0.6\columnwidth}\raggedright\strut
\strut
\end{minipage} & \begin{minipage}[t]{0.20\columnwidth}\raggedright\strut
¿No es igual?\strut
\end{minipage} & \begin{minipage}[t]{0.15\columnwidth}\raggedright\strut
215º\strut
\end{minipage} & \begin{minipage}[t]{0.05\columnwidth}\raggedright\strut
Si\strut
\end{minipage}\tabularnewline

\bottomrule
\caption{Caso de prueba CP-16.}
\end{longtable}

\subsection{Caso de prueba CP-17}

\textbf{Descripción:} Comprueba si se extraen correctamente los símbolos de un archivo DXF subido por el usuario y también comprueba que se han extraído todos, con la función  \texttt{upload\_symbols\_file()}

\textbf{Datos de entrada:} \texttt{Example\_simbology.dxf}


\begin{longtable}[]{@{}llll@{}}
\toprule
\begin{minipage}[b]{0.6\columnwidth}\raggedright\strut
Función\strut
\end{minipage} & \begin{minipage}[b]{0.20\columnwidth}\raggedright\strut
Condición\strut
\end{minipage} & \begin{minipage}[b]{0.15\columnwidth}\raggedright\strut
Salida Esperada\strut
\end{minipage} & \begin{minipage}[b]{0.05\columnwidth}\raggedright\strut
Ok\strut
\end{minipage}\tabularnewline
\midrule
\endhead
\begin{minipage}[t]{0.6\columnwidth}\raggedright\strut
test\_angle\_direction()\strut
\end{minipage} & \begin{minipage}[t]{0.20\columnwidth}\raggedright\strut
¿Incluye?\strut
\end{minipage} & \begin{minipage}[t]{0.15\columnwidth}\raggedright\strut
'Farola', 'Arbol', 'Vertice'\strut
\end{minipage} & \begin{minipage}[t]{0.05\columnwidth}\raggedright\strut
Si\strut
\end{minipage}\tabularnewline
\begin{minipage}[t]{0.5\columnwidth}\raggedright\strut
\strut
\end{minipage} & \begin{minipage}[t]{0.20\columnwidth}\raggedright\strut
¿No incluye?\strut
\end{minipage} & \begin{minipage}[t]{0.15\columnwidth}\raggedright\strut
'Casa', 'Banco'\strut
\end{minipage} & \begin{minipage}[t]{0.05\columnwidth}\raggedright\strut
Si\strut
\end{minipage}\tabularnewline
\begin{minipage}[t]{0.5\columnwidth}\raggedright\strut
test\_angle\_direction()\strut
\end{minipage} & \begin{minipage}[t]{0.20\columnwidth}\raggedright\strut
¿Es igual?\strut
\end{minipage} & \begin{minipage}[t]{0.15\columnwidth}\raggedright\strut
3\strut
\end{minipage} & \begin{minipage}[t]{0.05\columnwidth}\raggedright\strut
Si\strut
\end{minipage}\tabularnewline
\begin{minipage}[t]{0.5\columnwidth}\raggedright\strut
\strut
\end{minipage} & \begin{minipage}[t]{0.20\columnwidth}\raggedright\strut
¿No es igual?\strut
\end{minipage} & \begin{minipage}[t]{0.15\columnwidth}\raggedright\strut
0\strut
\end{minipage} & \begin{minipage}[t]{0.05\columnwidth}\raggedright\strut
Si\strut
\end{minipage}\tabularnewline
\bottomrule
\caption{Caso de prueba CP-17.}
\end{longtable}

\subsection{Caso de prueba CP-18}

\textbf{Descripción:} Comprueba el número de líneas o puntos que contiene un archivo de campo, con la función \texttt{upload\_topographical\_file()}

\textbf{Datos de entrada:} \texttt{Example\_topographic\_1.txt}


\begin{longtable}[]{@{}llll@{}}
\toprule
\begin{minipage}[b]{0.5\columnwidth}\raggedright\strut
Función\strut
\end{minipage} & \begin{minipage}[b]{0.20\columnwidth}\raggedright\strut
Condición\strut
\end{minipage} & \begin{minipage}[b]{0.15\columnwidth}\raggedright\strut
Salida Esperada\strut
\end{minipage} & \begin{minipage}[b]{0.05\columnwidth}\raggedright\strut
Ok\strut
\end{minipage}\tabularnewline
\midrule
\endhead
\begin{minipage}[t]{0.5\columnwidth}\raggedright\strut
test\_number\_lines\_topographical\_file()\strut
\end{minipage} & \begin{minipage}[t]{0.20\columnwidth}\raggedright\strut
¿Es igual?\strut
\end{minipage} & \begin{minipage}[t]{0.15\columnwidth}\raggedright\strut
42\strut
\end{minipage} & \begin{minipage}[t]{0.05\columnwidth}\raggedright\strut
Si\strut
\end{minipage}\tabularnewline
\begin{minipage}[t]{0.5\columnwidth}\raggedright\strut
\strut
\end{minipage} & \begin{minipage}[t]{0.20\columnwidth}\raggedright\strut
¿No es igual?\strut
\end{minipage} & \begin{minipage}[t]{0.15\columnwidth}\raggedright\strut
0\strut
\end{minipage} & \begin{minipage}[t]{0.05\columnwidth}\raggedright\strut
Si\strut
\end{minipage}\tabularnewline

\bottomrule
\caption{Caso de prueba CP-18.}
\end{longtable}

\subsection{Caso de prueba CP-19}

\textbf{Descripción:} Comprueba el número de líneas o puntos que contiene un archivo de campo que está vacío, con la función \texttt{upload\_topographical\_file()}

\textbf{Datos de entrada:} \texttt{Example\_topographic\_1.txt}


\begin{longtable}[]{@{}llll@{}}
\toprule
\begin{minipage}[b]{0.5\columnwidth}\raggedright\strut
Función\strut
\end{minipage} & \begin{minipage}[b]{0.20\columnwidth}\raggedright\strut
Condición\strut
\end{minipage} & \begin{minipage}[b]{0.15\columnwidth}\raggedright\strut
Salida Esperada\strut
\end{minipage} & \begin{minipage}[b]{0.05\columnwidth}\raggedright\strut
Ok\strut
\end{minipage}\tabularnewline
\midrule
\endhead
\begin{minipage}[t]{0.5\columnwidth}\raggedright\strut
test\_number\_lines\_topographical\_file()\strut
\end{minipage} & \begin{minipage}[t]{0.20\columnwidth}\raggedright\strut
¿Es igual?\strut
\end{minipage} & \begin{minipage}[t]{0.15\columnwidth}\raggedright\strut
0\strut
\end{minipage} & \begin{minipage}[t]{0.05\columnwidth}\raggedright\strut
Si\strut
\end{minipage}\tabularnewline
\begin{minipage}[t]{0.5\columnwidth}\raggedright\strut
\strut
\end{minipage} & \begin{minipage}[t]{0.20\columnwidth}\raggedright\strut
¿No es igual?\strut
\end{minipage} & \begin{minipage}[t]{0.15\columnwidth}\raggedright\strut
5\strut
\end{minipage} & \begin{minipage}[t]{0.05\columnwidth}\raggedright\strut
Si\strut
\end{minipage}\tabularnewline

\bottomrule
\caption{Caso de prueba CP-19.}
\end{longtable}

\subsection{Caso de prueba CP-20}

\textbf{Descripción:} Comprueba que el archivo DXF a generar tiene la versión correcta de CAD.

\textbf{Datos de entrada:} \texttt{Example\_topographic\_1.txt}


\begin{longtable}[]{@{}llll@{}}
\toprule
\begin{minipage}[b]{0.5\columnwidth}\raggedright\strut
Función\strut
\end{minipage} & \begin{minipage}[b]{0.20\columnwidth}\raggedright\strut
Condición\strut
\end{minipage} & \begin{minipage}[b]{0.15\columnwidth}\raggedright\strut
Salida Esperada\strut
\end{minipage} & \begin{minipage}[b]{0.05\columnwidth}\raggedright\strut
Ok\strut
\end{minipage}\tabularnewline
\midrule
\endhead
\begin{minipage}[t]{0.5\columnwidth}\raggedright\strut
test\_version\_cad()\strut
\end{minipage} & \begin{minipage}[t]{0.20\columnwidth}\raggedright\strut
¿Es igual?\strut
\end{minipage} & \begin{minipage}[t]{0.15\columnwidth}\raggedright\strut
AC1018\strut
\end{minipage} & \begin{minipage}[t]{0.05\columnwidth}\raggedright\strut
Si\strut
\end{minipage}\tabularnewline
\begin{minipage}[t]{0.5\columnwidth}\raggedright\strut
\strut
\end{minipage} & \begin{minipage}[t]{0.20\columnwidth}\raggedright\strut
¿No es igual?\strut
\end{minipage} & \begin{minipage}[t]{0.15\columnwidth}\raggedright\strut
AC1019\strut
\end{minipage} & \begin{minipage}[t]{0.05\columnwidth}\raggedright\strut
Si\strut
\end{minipage}\tabularnewline

\bottomrule
\caption{Caso de prueba CP-20.}
\end{longtable}